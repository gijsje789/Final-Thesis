%-------------------------------------------------------------------------------------------------
% This settings.tex contains settings required for *all* documents (reports, presentations, etc)
% Project or Report specific settings should go to their own settings files (eg CE/settings.tex)
% This file is included after the class definition and before project and report specific settings 
%-------------------------------------------------------------------------------------------------

%--------Useful packages (required by the example files, turn off if you do not use them)-------
\usepackage{babel}					% Add language specific support
%\usepackage{makeidx}				% Index support
%\usepackage[totoc,justific=RaggedRight]{idxlayout}	% Make last page of index balanced and add index to toc
\usepackage{caption}				% Provides means to style captions
%\usepackage{subcaption}				% Provides support for (sub)figures and (sub)tables
\usepackage{float}					% Improved interface for floating objects (eg figures, tables, ...)
\usepackage{enumitem}				% Add styling support to (enumerate) environments
\usepackage{listings}				% Allows (external) source files to be shown in a syntax highlighted way
%\usepackage{amsmath}				% Provides miscellaneous enhancements for documents containing formulas
\usepackage{datetime}				% Provides commands for displaying the current time
%\usepackage{etoolbox}				% Provides \AtBeginEnvironment command
\usepackage{eurosym}				% Defines \euro command to display euro symbols
%\usepackage{appendix}				% Makes it possible to modify appendix numbering
%\usepackage{longtable}				% Allows tables to span multiple pages
%\usepackage{units}					% Shows units (eg m/s) in a nice way
%\usepackage{ctable}				% Provides \ctable command for the typesetting of table and figure floats
%\usepackage{ccaption}				% Support continuation captions (eg multi-page tables)
%\usepackage{verbatim}				% Adds verbatim environment, in which texts are exactly copied to the output
%\usepackage{pdfpages}				% Include PDF pages/documents in the current document
\usepackage{color, colortbl}
\definecolor{Gray}{gray}{0.9}

\usepackage{tabularx}

\usepackage{xfrac}

\usepackage{pdflscape}

\usepackage{multicol}
\usepackage{multirow}

\usepackage[bottom]{footmisc}

\usepackage{makecell}

\usepackage[normalem]{ulem}

\usepackage{longtable}

%%%%%%%%%%%%%%%%%%%%%%%%%%%%%%%%%%%%%%%%%
%%%%%%% Acronyms %%%%%%%%%%%%%%%%%%%%%%%%
\usepackage{acro}

%\DeclareInstance{acro-title}{empty}{sectioning}{name-format =}

\DeclareAcronym{CT}{
	short 	= CT,
	long 	= Computed Tomography,
	class 	= abbrev
}

\DeclareAcronym{MRI}{
	short	= MRI,
	long	= Magnetic Resonance Imaging,
	alt		= MR,
	class 	= abbrev
}

\DeclareAcronym{SPECT}{
	short	= SPECT,
	long	= Single-Photon Emission Computed Tomography,
	class	= abbrev
}

\DeclareAcronym{PET}{
	short 	= PET,
	long	= Positron Emission Tomography,
	class	= abbrev
}

\DeclareAcronym{MPI}{
	short	= MPI,
	long	= Myocardial Perfusion Imaging,
	class	= abbrev
}

\DeclareAcronym{PET-MR}{
	short	= PET-MR,
	long	= PET-Magnetic Resonance,
	class	= abbrev
}

\DeclareAcronym{AIF}{
	short	= AIF,
	long	= Arterial Input Function,
	class	= abbrev
}

\DeclareAcronym{CZT}{
	short	= CZT,
	long	= Cadmium Zinc Telluride,
	class	= abbrev
}

\DeclareAcronym{CAD}{
	short	= CAD,
	long	= Coronary Artery Disease,
	class	= abbrev
}

\DeclareAcronym{VC} {
	short	= VC,
	long	= Vena Cava,
	class	= abbrev
}

\DeclareAcronym{PA}{
	short				= PA,
	long				= Pulmonary Artery,
	long-plural-form 	= Pulmonary Arteries,
	short-plural		= s,
	class	= abbrev
}

\DeclareAcronym{PV}{
	short				= PV,
	long				= Pulmonary Vein,
	long-plural 		= s,
	short-plural		= s,
	class	= abbrev
}

\DeclareAcronym{PMT}{
	short			= PMT,
	short-plural	= s,
	long			= Photomultiplier Tube,
	long-plural		= s,
	class			= abbrev
}

\DeclareAcronym{FWHM}{
	short			= FWHM,
	long			= Full Width at Half Maximum,
	class			= abbrev
}

\DeclareAcronym{HU}{
	short			= HU,
	long			= Hounsfield Unit,
	class			= abbrev	
}

\DeclareAcronym{MV}{
	short	= MV,
	long	= Maximal Vasodilation,
	class	= abbrev
}

\DeclareAcronym{ECTS}{
	short	= ECTS,
	long	= European Credit Transfer and Accumulation System,
	class	= abbrev
}

\DeclareAcronym{ROI}{
	short	= ROI,
	long	= Region of Interest,
	class	= abbrev
}

\DeclareAcronym{LV}{
	short	= LV,
	long	= Left Ventricle,
	class	= abbrev
}

\DeclareAcronym{RV}{
	short	= RV,
	long	= Right Ventricle,
	class	= abbrev
}

\DeclareAcronym{LA}{
	short	= LA,
	long	= Left Atrium,
	class	= abbrev
}

\DeclareAcronym{RA}{
	short	= RA,
	long	= Right Atrium,
	class	= abbrev
}
%%%%%%%%%%%%%%%%%%%%%%%%%%%%%%%%%%%%%%%%%%%%%%%%
%%%%%%%%%%%%%%%% END ACRONYM %%%%%%%%%%%%%%%%%%%

\usepackage{include/files/requirements}

\iffinalversion
	\usepackage[final]{include/files/notes}% Add note commands, [final] removes all notes from the document
	\usepackage[final]{include/files/rro}  % Add Rich Report Outline support, [final] removes all RRO output from document
\else
	\usepackage{include/files/notes}       % Add note commands
	\usepackage{include/files/rro}         % Add Rich Report Outline support
\fi

% Add wrongly (or unknown) hyphened words here (space separated and - at possible hyphenation positions):
%\hyphenation{}

%% Spacing possibilities for captions are available as well
% See captions.pdf for all options!
\captionsetup{font=small,labelfont=bf}

%% Center all figures by default
%% http://tex.stackexchange.com/questions/2651/should-i-use-center-or-centering-for-figures-and-tables
\makeatletter
\g@addto@macro\@floatboxreset\centering
\makeatother

%% Make use small font size in verbatim environment
% Note: AtBeginEnvironment is provided by etoolbox package
%\AtBeginEnvironment{verbatim}{\small}

%% Include verbatim in the subfigure env
% From: subfig.pdf, section 4.4
% <Uncomment if verbatim is required in subfloat>:
%\makeatletter
%\newbox\sf@box
%\let\orig@subfloat\subfloat
%\renewenvironment{subfloat}[2][]%
%{ \def\sf@one{#1}%
%  \def\sf@two{#2}%
%  \setbox\sf@box\hbox
%  \bgroup}%
%{ \egroup
%  \ifx\@empty\sf@two\@empty\relax
%    \def\sf@two{\@empty}
%  \fi
%  \ifx\@empty\sf@one\@empty\relax
%    \orig@subfloat[\sf@two]{\box\sf@box}%
%  \else
%    \orig@subfloat[\sf@one][\sf@two]{\box\sf@box}%
%  \fi}
%\makeatother
%% Uncomment till here
  
%% Automatically provide H option for floats
% Requires float package
% \floatplacement{figure}{H} 
% \floatplacement{table}{H} 

%% abbreviation making
\newcommand{\abbr}[1]{(\textit{#1})}

%%lstlisting settings
\lstset{	%aboveskip=20pt,%
		numbers=none, %no line numbers
%		numbers=left, %show line numbers
		numberstyle=\tiny,%
		frame=single,%
		frameround={t}{t}{t}{t},%
		numbersep=5pt,%
%		language=C,% (default) code language in the document
		captionpos=b,%
		xleftmargin=2em,
		framexleftmargin=1.5em,
		xrightmargin=2em,
		framexrightmargin=1.5em,
		morecomment=[s][\itshape]{<}{>}, %also define <> as comment
		morecomment=[s][\itshape]{[}{]} %also define [] as comment
}

%lstinline with empty language definition
\lstdefinelanguage{empty}{}
\newcommand{\mylstinline}[1]{{\lstinline[language=empty]{#1}}}

% Default value of top separator (empty space) of lists
\setlist{topsep=4pt}

%% Don't show warnings like: ``PDF inclusion: found PDF version <1.x>, but at most version <1.4> allowed
% Uncomment if you experience these kind of warnings 
%\ifpdf
%	\pdfminorversion=6 
%\fi