\chapter{Literature}
\label{ch:literature}
\rrot{Read available literature}
\rrot{Write literature review to more accurately define research questions}
\rroi{D-SPECT literature?}
\rroi{Discuss division of work}

\section{MR based phantoms}
\section{CT based phantoms}
\cite{suzuki2017quantitative} designed a straight-forward Computed Tomography phantom that uses a dry-type hemodialyzer where the dialysate space is pressurised with air to prevent the perfusate from leaving the dialysate fibres. The authors varied the dose in order determine the effects on the perfusion indices. They maintained a constant volumetric flow, Q, and concluded that the perfusion indices are susceptible to dose conditions. Furthermore, the straight forward phantom does not resemble the human brain, which caused problems in certain programs, and that the capillary possessions is much greater than in clinical situations. This may ultimately compromise the reliability of the phantom to mimic clinical situations. \cite{hashimoto2018effect} used the same phantom but with a commercially synthetic bone layer such that quantification software recognises the phantom as a human head. Instead of varying the dose, the contrast injection protocol and the scanning interval is varied based on their hypothesis that it would increase the quantitative accuracy. However, they concluded that they are independent factors when using the b-SVD algorithm. The phantom, that both papers uses, does not simulate contrast uptake by the heart tissue.
\section{PET/SPECT based phantoms}