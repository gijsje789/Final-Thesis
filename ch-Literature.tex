\chapter{Literature}
\label{ch:literature}
\rrot{Read available literature}
\rrot{Write literature review to more accurately define research questions}
\rrot{D-SPECT literature?}
\rroi{Discuss division of work}
\rroi{Dialysis tube mimics capillaries and not tissue?}
\rroi{Mathys reported that no literature was found that manufacturers calibrate their scanners.}

\section{Phantoms}
\subsection{Magnetic Resonance}
\rrot{Shift the limitations to later section}
\cite{noguchi2007quantitative} developed a simple phantom that consists of a syringe, diameter of 40mm, beads, and tubes, diameter of 4mm. The perfusate, 0.1mM GD-DTPA doped 8L water solution, flows through the beads, to disturb the flow, and then perfuses through parallel tubes, to prevent cross-current. The perfusate is tagged at the beads while the images are taken at 5 parallel planes, perpendicular to the tubes. The actual measured flow is the average obtained from the parallel planes. 

\cite{ebrahimi2010microfabricated} created a phantom using microfabrication to create a microvasculature on 4-inch ($\approx$ 10cm) silicone wafers. The microvasculature is build up from four blocks, containing 4x2 (RxC) cells. Each cell is build up from 100 "features" separated by 25$\mu$m tracks. These tracks provide many different paths for perfusate to flow, effectively simulating capillaries in tissue. A $300\mu L$ bolus of a distilled water solution containing 25mM/L of Manganese.

\cite{wang2010flow} used a heamodialysis filter connected to a nonpulsatile pump. A static water phantom was placed next to the haemodialysis filter to show that it cancels out between tag (with magnetic labelling) and control (no magnetic labelling) images.

\cite{anderson2011semipermeable} extracted hollow fibres from haemodialysis filters to create their own single-fibre and multi-fibre phantom. Similar to some standard haemodialysis filters, their phantoms have access to both the extracellular space (i.e. the fluid outside of the fibres) and intracellular space (i.e. the fluid inside the fibres). As the name suggests, their single-fibre phantom consists of an individual fibre placed inside a capillary tube. The multi-fibre phantom consists of a variable amount of fibres that are placed in a heat-shrink tube. A four-way valve switched between the main perfusate and a aquesous solution of 0.2 mM GD-BOPTA.

\cite{chiribiri2013perfusion} developed a four chambered anatomic phantom that resembles the heart of a 60kg person. The four chambers correspond to the four chambers of the heart, sized to match the physiological size. In addition, a \ac{VC}, \ac{PA}/\ac{PV} combination, and an aorta are present in the phantom. Contrast is injected in the same manner as is performed in patients; in a vein. In the phantom, the contrast is injected directly into the vena cava via a three-way stopcock. The contrast moves through the phantom's right atrium, right ventricle then via the \ac{PA}/\ac{PV} to the left chamber and finally to the left ventricle. The phantom is not capable of simulating the contrast's behaviour in the lungs since the \ac{PA} is directly connected to the \ac{PV}. Two myocardial chambers (one active and one control), the \ac{VC}, \ac{PA}, \ac{PV} and the aorta are in the imaging plane where the proximal part of the aorta, where the aorta branches to the myocardium, is used for the \ac{AIF}. \textbf{chriibiri The phantom is unable to simulate the diffusion of contrast into heart tissue or the interstitial space, as admitted by the authors and confirmed by \cite{otton2013direct, o2017effect}. Furthermore, \cite{chiribiri2013perfusion} mentioned that the blood flow resistance is lower than in patients due to its complexity.}

\cite{otton2013direct} used the same phantom to compare \ac{CT} against \ac{MRI}.\textbf{ Their findings are similar; the contrast curves represent those obtained from clinical trials. Since the phantom can be used in a clinical \aca{MRI} scanner, the gap between phantom and clinical studies is decreased.} In addition to the previous authors, \cite{o2017effect} used a water-filled torso phantom to ensure more anatomically correct image in \ac{PET-MR}. \textbf{However, it is still unable to mimic respiratory or cardiac motion \citep{o2017feasibility}.} 

\subsection{Computed Tomography}
\cite{teslow1991x} developed a cylindrical perfusion phantom, shaped like the left ventricle of a dog. It has a 6.5cm outer diameter, 4.5cm inner diameter, and a length of 6.5cm. The authors specifically used methyl methacrylate plastic since it gives similar radiographic image as tissue and blood. Additionally, a solid plastic cylinder, placed in the centre, is used to attenuate x-ray beams similar to the attenuation of a blood-filled dog's ventricle. The capillaries are simulates by means of different sized nylon balls of 0.318, 0.476, and 0.635cm. The smallest balls are packed near the outlet, while the medium sized balls are packed at the inlet, and the largest balls are placed in between. The authors do not go into detail on the used contrast agent, other than  10ml of a radio-opaque indicator is injected for one second.

\cite{driscoll2011development} developed a 10cm long, with a diameter of 5cm, phantom with two inputs, only one is used, and two outputs. The capillaries are simulated by means of a vinyl tube where mass can be exchanged with the main cylinder via sets of small holes on either side of the cylinder. The output of the vinyl tube and the output of the shell combine, and feeds back through the imaging area to validate outward going flow, which should be the same as the input flow. Their \ac{AIF}, however, is created by means of a programmable pump which injects in the form of a typical clinical \ac{AIF} rather than having a system-based \ac{AIF}. The Visipaque\textsuperscript{TM} (iodixanol) 270 mgI/ml is injected into a blood-mimicking fluid consisting of 40\% glycerol and 60\% water.

\cite{ganguly2012vitro} developed an interestingly, slightly different phantom than other;  a linearly moving phantom. The phantom itself is a cylinder, 1.9cm inner diameter, with a length of 32.2cm. It contained 64 different compartments of 0.5cm in height separated by a 0.5mm thick carbon fibre wall. Every compartment had a single opening where contrast, 300mgI/Omnipaque, is injected. The concentration of contrast that is injected in subsequent compartments, resemble a sinusoidal signal. This phantom is placed on a linear motor which moves the phantom parallel to the patient's bed. The author's goal is to compare and determine the temporal accuracy of the imaging system. To simulate the attenuation of the head, a 15cm cylindrical, water-filled, phantom was placed around the perfusion phantom. 

\cite{mathys2012phantom} developed a similar phantom that consists of two cylinders, a smaller (4cm diameter) inside a larger (11cm diameter) cylinder. Water flows via four tubes into the inner cylinder where it flows outward into the larger cylinder that contains 1.5mm polyoxymethylene globes as tissue replacement. The larger cylinder is drained by four holes. 2mL of contrast, Accupaque 300, is injected followed by 15 of saline at 5mL/s using a double-head injector.

\cite{boese2013performance} developed a cylindrical phantom for brain perfusion measurements in a C-arm. Their phantom utilises a combination of large arteries, smaller arteries, and a sinter board for the capillaries. The main artery splits into smaller arteries which in turn splits into sixteen even smaller ones that connect to the sinter board. The upper two arteries have an inner diameter of 1.7mm representing the carotid arteries and the lower two arteries have an inner diameter of 1.0mm representing vertebral arteries. The authors used a very specific contrast injection protocol: relatively large pre-injection of 21mL NaCl, followed by a variable amount of Imeron 400, and ended by a 6mL post-injection of NaCl, all at 6mL/s.

\cite{suzuki2017quantitative} designed a straight-forward \ac{CT} phantom that uses a dry-type haemodialyser with a pressurised dialysate space to prevent the perfusate from leaving the hollow fibres. The authors varied the dose in order determine the effects on the perfusion indices. They maintained a constant volumetric flow, Q, and concluded that the perfusion indices are susceptible to dose conditions. \textbf{Furthermore, the straight forward phantom does not resemble the human brain, which caused problems in certain programs, and that the capillary possessions is much greater than in clinical situations. This may ultimately compromise the reliability of the phantom to mimic clinical situations.}

\cite{hashimoto2018effect} used the same phantom in combination with a commercially synthetic bone layer such that quantification software recognises the phantom as a human head. Instead of varying the dose, the contrast injection protocol and the scanning interval are varied based on their hypothesis that it would increase the quantitative accuracy. However, they concluded that they are independent factors when using the b-SVD algorithm. \textbf{The phantom, that both papers uses, does not simulate contrast uptake by the heart tissue.}

\subsection{Ultrasound}
\rrod{Veltmann phantom}

\cite{veltmann2002design} designed a flow phantom that consists of a high- and low flow circuit. The high flow circuit consists of a \textit{heated} reservoir flowing into a haemodialysis cartridge, which filters any residue micro-bubbles (contrast agent) and removes air bubbles, before entering a second haemodialysis cartridge, the perfusion cartridge. Perfusate that does not enter the capillaries is returned to the reservoir passing a variable resistance. The perfusate that does enter the capillaries of the perfusion cartridge, is controlled by a gear pump, which simultaneously acts as a variable flow resistance for the low flow circuit. After the gear pump, a third haemodialysis filter filters the microbubbles from the perfusate. The authors performed two different experiments, one with an unmodified haemodialysis filter and one with a haemodialysis filter that has the majority of the lower capillaries glued shut. The contrast agent tends to float, especially in the low flow circuit. By decreasing the number of perfused capillaries, the flow is made more homogeneous and avoids attenuation in the lower areas. Both \cite{sakano2015power} and \cite{lohmaier2004vitro} use this phantom setup.

\cite{kim2016efficiency} performs perfusion experiments using ultrasound without adding any contrast. Similar to the \ac{CT} and \aca{MRI} phantoms, a dialysis tube is used to mimic human capillaries. The dialysis tube is submerged in water and part of the plastic case was removed, replaced by a latex foil as proposed by \cite{veltmann2002design}, such that it creates an acoustic window. More interestingly, \cite{kim2016efficiency} use a secondary, 1Hz, peristaltic pump to simulate cardiac motion. \cite{gauthier2011perfusion} uses a peristaltic pump after a renal dialysis cartridge to create a pulsatile, but constant, flow. They do not use a secondary pump for the extracellular space.

\subsection{Positron Emission Tomography / Single-Photon Emission Computed Tomography}
Although the phantoms are not specifically designed for \ac{PET} or \ac{SPECT} scanning, the previously mentioned phantoms can be an inpsirational source for \ac{PET}/\ac{SPECT} phantoms. The different technology requires a new approach to some (or many) of the materials used.

\subsection{Phantom discussion}

\section{Physiology}


\section{Technology}
\subsection{SPECT}
The imaging method in a typical \ac{SPECT} scanner are scintillator-based gamma cameras, also known as Anger cameras. Gamma cameras use a scintillator to "transduce" gamma radiation, originating from an injected tracer, to photons. Part of these photons are directed towards a photocathode. If a quantum of light hits the photocathode, which is coated with a photosensitive coating, electrons are emitted due to the photoelectric effect. These electrons travel through a \acp{PMT} and hitting series of dynodes, which in turn trigger secondary emission effectively multiplying the number of electrons travelling through the tube. Electrons hitting the last dynode, which is known as the anode, cause a current pulse which can be detected by measuring equipment. It is proportional to the amount of gamma ray photons entering the scintillator\citep{CZTTech2009}.

\subsection{Digital SPECT}
Developments in imaging systems gave rise to the digital \ac{SPECT} scanner. In contrast to the analogue Anger cameras, the digital \ac{SPECT} scanner utilises a direct conversion semiconductor: \ac{CZT}. \cite{wagenaar2004cdte} used \ac{CZT} to develop pixelated detector units which could then be used for medical imaging. In a recent study, it is shown that a digital \ac{SPECT} scanner, using multiple pixelated \ac{CZT} detectors, showed significant improvements in image sharpness and contrast\citep{goshen2018feasibility}. These detector units do not require any \acp{PMT} and thus allow for a more compact and flexible design \citep{erlandsson2009performance}. The D-SPECT scanner, a digital \ac{SPECT} scanner developed by Spectrum Dynamics\footnote{https://www.spectrum-dynamics.com/}, offers improvements in sensitivity and energy resolution \citep{SpectDynam2018} over Anger camera systems. However, these digital systems are relatively new and require proper validation to convince medical personnel of its value.

\subsection{Scanner comparison}
As is previously mentioned, there are various types of scanners that use different techniques, \ac{CT} \ac{MRI}, or Scintigraphy based (\ac{SPECT}/\ac{PET}) scanners. In cardiology, the \ac{SPECT} scanner is widely employed for coronary and myocardial perfusion measurements \citep{rahmim2008pet}. It is known that \ac{PET} scans are generally more expensive \citep{hlatky2014economic, RadioPead2018}. \cite{hlatky2014economic} followed patients for two years, recording the costs and concluded that \ac{PET} costs are 22\% higher than the costs for \ac{SPECT} for patients with suspected \ac{CAD}.