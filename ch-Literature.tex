\chapter{Literature}
\label{ch:literature}
\rrot{Read available literature}
\rrot{Write literature review to more accurately define research questions}
\rroi{D-SPECT literature?}
\rroi{Discuss division of work}

\section{Phantoms}
\subsection{Magnetic Resonance}
\cite{chiribiri2013perfusion} use a four chambered phantom, to simulate the four chambers of the heart. In addition, a vena cava, pulmonary artery/vein combination, and an aorta are present in the phantom. Contrast is injected in the same manner as is performed in patients; in a vein. In the phantom, the contrast is injected directly into the vena cava via a three-way stopcock. The contrast moves through the phantom's right atrium, right ventricle then via the pulmonary artery/vein to the left chamber and finally to the left ventricle. The phantom is not capable of simulating the contrast's behaviour in the lungs since the pulmonary artery is directly connected to the pulmonary vein. Two myocardial chambers, the vena cava, pulmonary artery, pulmonary vein, and the aorta are in the imaging plane where the proximal part of the aorta is used for the AIF. The phantom is unable to simulate the diffusion of contrast into heart tissue or the interstitial space, as admitted by the authors and confirmed by \cite{otton2013direct, o2017effect}. Furthermore, \cite{chiribiri2013perfusion} mentioned that the blood flow resistance is lower than in patients due to its complexity. 

\cite{otton2013direct} used the same phantom to compare CT against MRI. Their findings are similar; the contrast curves represent those obtained from clinical trials. Since the phantom can be used in a clinical MR scanner, the gap between phantom and clinical studies is decreased. In addition to the previous authors, \cite{o2017effect} used a water-filled torso phantom to ensure more anatomically correct image in PET-MR. However, it is still unable to mimic respiratory or cardiac motion.
\subsection{Computed Tomography}
\cite{suzuki2017quantitative} designed a straight-forward Computed Tomography phantom that uses a dry-type haemodialyzer where the dialysate space is pressurised with air to prevent the perfusate from leaving the dialysate fibres. The authors varied the dose in order determine the effects on the perfusion indices. They maintained a constant volumetric flow, Q, and concluded that the perfusion indices are susceptible to dose conditions. Furthermore, the straight forward phantom does not resemble the human brain, which caused problems in certain programs, and that the capillary possessions is much greater than in clinical situations. This may ultimately compromise the reliability of the phantom to mimic clinical situations. \cite{hashimoto2018effect} used the same phantom but with a commercially synthetic bone layer such that quantification software recognises the phantom as a human head. Instead of varying the dose, the contrast injection protocol and the scanning interval is varied based on their hypothesis that it would increase the quantitative accuracy. However, they concluded that they are independent factors when using the b-SVD algorithm. The phantom, that both papers uses, does not simulate contrast uptake by the heart tissue.
\subsection{Positron Emission Tomography / Single-Photon Emission Computed Tomography}

\section{SPECT technology}
The imaging method in a typical SPECT scanner are scintillator-based gamma cameras, also known as Anger cameras. Gamma cameras use a scintillator to "transduce" gamma radiation, originating from an injected tracer, to photons. Part of these photons are directed towards a photocathode. If a quantum of light hits the photocathode, which is coated with a photosensitive coating, electrons are emitted due to the photoelectric effect. These electrons travel throught Photomultiplier tubes (PMTs) and hitting series of dynodes, which in turn trigger secondary emission effectively multiplying the number of electrons travelling through the tube. Electrons hitting the last dynode, which is known as the anode, cause a current pulse which can be detected by measuring equipment. It is proportional to the amount of gamma ray photons entering the scintillator\citep{CZTTech2009}.

Developments in imaging systems gave rise to the Digital SPECT scanner. In contrast to the analogue Anger cameras, the D-SPECT scanner utilises a direct conversion semiconductor: Cadmium Zinc Telluride (CZT). \cite{wagenaar2004cdte} used CZT to develop pixelated detector units which can be used for medical imaging. In a recent study, it is shown that a Digital SPECT scanner, using multiple pixelated CZT detectors, showed significant improvements in image sharpness and contrast\citep{goshen2018feasibility}. These detector units do not require PMTs and thus allow for a more compact and flexible design \citep{erlandsson2009performance}. The D-SPECT scanner, developed by Spectrum Dynamics\footnote{https://www.spectrum-dynamics.com/}, offers improvements in sensitivity and energy resolution \citep{SpectDynam2018} over Anger camera systems. However, these digital systems are relatively new and require proper validation to convince medical personnel of its value.

\section{Comparison to physiology}