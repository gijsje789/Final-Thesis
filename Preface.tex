\chapter*{Preface}

\vskip-10pt
The system requirements specify all the requirements for the myocardial perfusion phantom. These requirements are based on research and interviews with stakeholders.

\vskip10pt
G.J. (Gijs) de Vries\\
Enschede, 7\textsuperscript{th} of January 2019

\vskip10pt
\section*{Version}
\begin{longtable}{|c|p{0.64\linewidth}|l|}
		\multicolumn{1}{l}{\textbf{Requirement}} & \multicolumn{1}{c}{\textbf{Old description}} & \multicolumn{1}{c}{\textbf{Date}} \endhead
		\hline
		R0.1 & Initial version. Discussed in progress meeting of 2019/01/15. & 2019/01/15 \\
		R0.2 & 
			Added following items: & 2019/01/16 \\
		 	& \hspace{0.5cm}\textbullet TR-PR02, TR-PR03, TR-PR04, TR-PR05, TR-PR06, TR-PR07, TR-PR08, TR-PR09, TR-PR10 & \\
		 	& \hspace{0.5cm}\textbullet TR-ER02, TR-ER03 & \\
		 	& \hspace{0.5cm}\textbullet TR-IC04, and TR-IC05 & \\
		 	& \hspace{0.5cm}\textbullet Added appendices C \& D. & \\
		 	R0.21 & & 2019/01/18 \\
		 	R0.22 & Added following items: & 2019/01/20 \\
		 	& \hspace{0.5cm}\textbullet TR-IC03 A) through E). & \\
		 	& \hspace{0.5cm}\textbullet TFR-SIM04 A) through E) & \\
		 	& \hspace{0.5cm}\textbullet TR-PR02 A) through E). & \\
		 	R0.23 & Added following items: & 2019/01/23 \\
		 	& \hspace{0.5cm}\textbullet TFR-SIM05 A) through C). & \\
		 	R0.24 & Modified: & 2019/01/28 \\
		 	& \hspace{0.5cm}\textbullet Section \ref{sec:concept_oper}, to correspond to interview at ZGT. & \\
		 	& Added following items: & \\
		 	& \hspace{0.5cm}\textbullet Figure \ref{fig:sim_heart}, \ref{fig:segment_heart}, and \ref{fig:segment_supply} & \\
		 	& Removed following items: & \\
		 	& \hspace{0.5cm}\textbullet FR08 (combined with FR07) & \\
		 	& Inserted following items: & \\
		 	& \hspace{0.5cm}\textbullet TFR-SIM04 B) \& C). Other requirements are shifted down. & \\
		 	R0.25 & Modified: & 2019/01/29 \\
		 	& \hspace{0.5cm}\textbullet Section \ref{sec:what_perf}, rephrased. & \\
		 	& Removed following items: & \\
		 	& \hspace{0.5cm}\textbullet TFR-SIM04 E), combined with TFR-SIM04 D), AIF initially in left atrium but alternatively in left ventricle. & \\
		 	& Added following items: & \\
		 	& \hspace{0.5cm}\textbullet TFR-GF09. & \\
		 	R0.26 & Textual (argumentative) requirements are separated from the quantitative requirements. & 2019/01/30 \\
		 	& Modified following items: &  \\
		 	& \hspace{0.5cm}\textbullet Caption of figure \ref{fig:funcarch} to make it clear that it is not definitive. & \\
		 	& \hspace{0.5cm}\textbullet Business model, rephrased and added business cases as discussed in work meeting of January 29, 2019. & \\
		 	& Added following items: & \\
		 	& \hspace{0.5cm}\textbullet TFR-GFQ03, TFR-GFQ04, TFR-GFQ09, TFR-GFQ10. & \\
		 	R0.27 & Textual (argumentative) requirements are separated from the quantitative requirements. & 2019/01/31 \\
		 	& Removed following items: & \\
		 	& \hspace{0.5cm} \textbullet TR-PRT03 A) and D), double requirements. & \\
		 	R0.28 & Textual (argumentative) requiremens are separated from the quantitative requirements. & 2019/02/01 \\
		 	& Added following items: & \\
		 	& \hspace{0.5cm} \textbullet TR-SQ03 & \\
		 	& \hspace{0.5cm} \textbullet TFR-SIMT04 & \\
		 	& Removed following items: & \\
		 	& \hspace{0.5cm} \textbullet TR-ERT03, the patient chair is in supine (flat) position and should provide enough space for the set-up. & \\
		 	R0.29 & Renamed environmental requirements to external requirements; the requirements that are specified from the outside of the system.  & 2019/02/04 \\
		 	& Added following items: & \\
		 	& \hspace{0.5cm} \textbullet TR-PRQ07 & \\
		 	R0.210 & Added technical block diagram. & 2019/02/05 \\
		 	& Removed following items: & \\
		 	& \hspace{0.5cm} \textbullet TFR-ICT05 B), tracer injection parameters are now specified in TFR-ICQ04, TFR-ICQ05, and TFR-ICQ06. &\\
		 	& Added following items: & \\
		 	& \hspace{0.5cm} \textbullet TFR-ICQ04, TFR-ICQ05, TFR-ICQ06. & \\
		\hline
\end{longtable}

\newpage

\section*{Changelog}
%\begin{table} [H]
%\caption{ Table of changed requirements }
	\begin{longtable}{|c|p{0.5\linewidth}|p{0.26\linewidth}|}
		\multicolumn{1}{l}{\textbf{Requirement}} & \multicolumn{1}{c}{\textbf{Old description}} & \multicolumn{1}{l}{\textbf{Change reason}}  \endhead
		\hline
		TR-IC01 & A variable amount of contrast can be injected. & Rephrased. \\
		TFR-SIM01 & An \ac{AIF} must be extractable from either the aorta or the left ventricle chamber. & The AIF, in the D-SPECT software, is taken from the left ventricle. This requirement is moved to TFR-SIM04. \\
		TFR-SIM04 & Multiple chambers, or areas, should be present, such that ischaemic and non-inschaemic tissue can be visualised simultaneously. Typical software divide the heart into 17 chambers. & Rephrased due to misunderstanding of the 17 section model.\\
		TR-PR10 & The phantom's chambers must match the dimensions of an average human heart, between 60-90x30-50x60-90mm [LxWxD] & Sizes are specified for the ventricles. \\
		TR-PR02 & The phantom must be anatomically correct; four heart chambers, myocardium around the chambers, arrow shaped bottom. & Rephrased after interview at ZGT. \\
		TFR-SIM05 & Phantom's compartment model should match the currently practised protocol. Does the tracer diffuse, is it trapped in tissue et cetera. & Rephrased and linked to contrast section. \\
		FR03 & The high flow should be suitable for an \ac{AIF}, either in a ventricle chamber or an aorta depending on the clinical software.  & The D-SPECT software extracts the AIF in the left atrium.\\
		FR04 & Cardiac defects should be simulated such that the complex relation between stenotic and non-stenotic arteries is modelled. & Rephrased. \\
		FR05 & The phantom must be able to visualise both control and stenotic areas, similar to clinical scans. & Rephrased, it should be compatible with the 17 segment model. \\
		FR06 & The phantom must initially simulate the compartment model typically used in clinical scans, but be flexible enough such that other compartment models are achievable. & Rephrased to be more specific.\\
		FR07 & The contrast agent should be equivalent to that used in clinical scans. & Rephrased and combined with FR08 to be more global. \\
		TFR-SIM04 A) & The three coronary arteries should be present (RCA, LAD, LCx) and connected to a myocardium. & Rephrased to make it more clear. \\
		TFR-SIM04 F) & The myocardium has a longitudinal cross-sectional shape of a horseshoe. & Rephrased to be more specific. \\
		TFR-SIM04 G) &  The myocardium has a transverse cross-sectional shape of a circle. & Rephrased to be more specific. \\
		TFR-SIM04 D) & An ROI can be taken in the left ventricle. & Combined with TFR-SIM04 E), AIF is taken in left atrium. If it has poor results, the AIF's ROI can be moved to the left ventricle. \\
	TFR-SIM04 E) & An AIF can be taken from the left atrium. & Removed, combined with TFR-SIM04 D).\\
	TR-PR02 A) & In correspondence with requirements TFR-SIM04 D) and E) & TFR-SIM04 requirements were modified, therefore TR-PR02 is modified in accordance. \\
	TFR-GF03 & Minimum achievable upper limit of myocardial perfusion is 300 mL/min/100g. & Added more specificity for stress perfusion.  \\
	TFR-GF04 & Minimum achievable lower limit of myocardial perfusion is 60 mL/min/100g. & Added more specificity for rest perfusion. \\
	FR05 & The phantom must be able to visualise (and measure) the 17-segment cardiac model. & 17 active segments will be too much for the initial version. \\
	FR03* & The high flow should be suitable for an \ac{AIF} extracted from the left atrium. & Rephrased to be more specific.\\
	FR07 & The contrast protocol must be equivalent to that used in clinical scans with D-SPECT. & For SPECT, the terminology is "tracer" in stead of "contrast". \\
	TR-PR01 & The phantom, and its set-up, must fit on the D-SPECT's chair. & The phantom itself must fit on the chair and in the imaging area. However, the set-up surrounding the phantom (flow generators, measurement systems et cetera) do not necessaries \\
	TFR-ICT04 & Tracer injection is reproducible. & Tracer injection is reproducible using an infusion pump. Too much variation exists when tracer is injected manually.\\
	TR-PRT01 & The phantom is to be placed inside the QRM TRX-116, see TR-PRQ01. & The phantom's left ventricle is to be placed inside the thorax phantom as opposed to the entire phantom.\\
	TR-PRT02 & The phantom must fit on the D-SPECT seating in the imaging area. & The left ventricle must be in the imaging area as opposed to the entire phantom.\\
	TFR-GFT01 & A constant flow is to be generated, i.e. non-pulsatile. & Flow must be constant and variable.\\
		\hline
	\end{longtable}
%\end{table}