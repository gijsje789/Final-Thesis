\chapter{Discussion \& conclusion}
\section{Discussion}
The majority of the requirements have been verified, or verified under certain conditions. Some requirements have been failed, some have not been tested and the rest are no longer applicable.

The two most important requirements that have not been met, are SIMT05 B) and PRT05. SIMT05 B) states that the tracer should be trapped to approximately 1.2\% of the administered activity. The R3 chamber design, using tubes with 1mm holes, does in a way trap the tracer, but only temporarily. This prevents the software from accurately calculating the myocardial flow since it is based on a 2-compartment model. PRT05 states that the phantom must be easily be cleared of air bubbles. During experiments with the first myocardial chamber design, the air bubbles were easily removed by tilting the phantom. However, the R3 chamber design was not cleared of air bubbles properly despite being able during university experiments. The lack of venting may be caused by the work flow at the hospital (less manoeuvre room, less time spent trying to vent) or by it being a design flaw. New experiments with the current set-up, and with identical configuration, should show whether it is a design flaw or work flow error.

Nevertheless, the main goal of the first phase phantom, was to determine if a more anatomically correct perfusion phantom can be designed and realised that is compatible with clinical practice. 4DM recognised enough features to allow reconstruction of the images.

\section{Conclusion}
Despite its flaws, the first phase phantom showed that the phantom can be used in conjunction with the 4DM clinical software. The second phase will require a redesign of various parts of the flow set-up and/or phantom.