\chapter{Technical}

\section{Physical requirements}
Size, weight et cetera.
\rrot{Determine size of D-SPECT seating}
\rrot{Determine shape of D-SPECT head}
\begin{table} [h]
\caption{Functional requirements}
\label{tab:physres}
This table summarises the physical requirements.
\begin{tabular}{l|p{120mm}|}
	\makecell[l]{Requirement \\ number} & \multicolumn{1}{c}{Description}\\
	\hline
	PR01 & The phantom, and its set-up must fit in the D-SPECT scanner. \\ 
	\cline{2-2}
\end{tabular}
\end{table}

\section{Environmental requirements}
\rrot{Determine seating height for external support.}
In what environment is the system operating.

\section{External interfaces}

\section{Requirements}
techincal details, operating environments, constraints et cetera.

\section{System qualities}
Define the quality of the system: such as reliability, availability, serviceability, security, scalability, maintainability.

\section{Constraints and Assumptions}
Design constraints that have been imposed and assumptions that have been made by the requirements engineering team when gathering and analyzijng the requirements.
