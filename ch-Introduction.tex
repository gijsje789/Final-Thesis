\chapter{Introduction}
\label{ch:Intro}

\rrod{Read into background information on D-SPECT}
\rrod{Write global background information}
\rrod{Introduce the rest of the document}
\rrod{Assignment was for dynamic SPECT scanning, but is that the same as using the D-SPECT? The D-SPECT can scan dynamically, and is available in ZGT}
\rrod{Too much SPECT detail in introduction? Moved to literature}
\rrot{Give arguments why to choose SPECT}

There are various types of scanners that use different techniques. Examples are Computed Tomography (CT), Magnetic Resonance Imaging (MRI), or Scintigraphy (SPECT/PET) scanners. In cardiology, the SPECT scanner is widely employed for coronary and myocardial perfusion measurements \citep{rahmim2008pet}. It is known that PET scans are generally more expensive \citep{hlatky2014economic, RadioPead2018}. \cite{hlatky2014economic} followed patients for two years, recording the costs and concluded that PET costs are 22\% higher than the costs for SPECT for patients with suspected Coronary Artery Disease (CAD).

\section{Document overview}
\label{sec:doc_overview}
\rrod{Update in correspondence with meeting december 10}
The project plan consists of a literature review of existing myocardial perfusion phantoms and their comparison to human physiology, and more extensive information on D-SPECT scanners (their technical background, limitations, and so forth). The literature is followed by the research methodology containing the research questions and goals of the project. The detailed planning is the last section of the project plan stating workdays and -weeks, off-days, deadlines, and meetings.