\chapter{Introduction}
\label{ch:Intro}

\Ac{MPI}, or, simply put, the imaging of the blood flow in the heart muscle, plays an important role in diagnosing heart failure or detecting \ac{CAD}. Imaging systems like \ac{CT}, \ac{MRI}, \ac{SPECT}, or \ac{PET} can visualise a (radioactive) contrast bolus in the supplying arteries and in underlying myocardial tissue, whose flow can give an indication of narrowed or blocked blood vessels.

Many variations in the visualisation process of myocardial perfusion, including variations in hard- and software, can (significantly) influence the outcome and in turn have consequences for patient treatment. These variations need to be validated against a well-known baseline.

A myocardial perfusion phantom will be developed that is able to simulate the blood flow in the heart muscle, i.e. the myocardium, and is able to mimic cardiac defects like (significant) stenosis.

\section*{Document overview}
\label{sec:doc_overview}
\rrot{This section}

\section*{Abbreviations}
\begin{multicols}{2}
	\printacronyms[include-classes=abbrev, name=Abbreviations, heading=none]
\end{multicols}