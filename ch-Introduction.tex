\chapter{Introduction}
\label{ch:Intro}

\rrod{Read into background information on D-SPECT}
\rrod{Write global background information}
\rrod{Introduce the rest of the document}
\rrod{Assignment was for dynamic SPECT scanning, but is that the same as using the D-SPECT? The D-SPECT can scan dynamically, and is available in ZGT}
\rrod{Too much SPECT detail in introduction? Moved to literature}
\rrod{Give arguments why to choose SPECT, in SR}

\Ac{MPI}, or, simply put, the imaging of the blood flow in the heart muscle, plays an important role in diagnosing heart failure or detecting \ac{CAD}. Imaging systems like \ac{CT}, \ac{MRI}, \ac{SPECT}, or \ac{PET} can visualise a (radioactive) contrast bolus in the supplying arteries and in underlying myocardial tissue, whose flow can give an indication of narrowed or blocked blood vessels.

Many variations in the visualisation process of myocardial perfusion, including variations in hard- and software, can (significantly) influence the outcome and in turn have consequences for patient treatment. These variations need to be validated against a well-known baseline.

\section*{Document overview}
\label{sec:doc_overview}
\rrod{Update in correspondence with meeting december 10}
The project plan consists of a literature review of existing myocardial perfusion phantoms, their comparison to human physiology, and a discussion between the different types of scanners. The literature is followed by the research methodology containing the research questions and goals of the project. The detailed planning is the last section of the project plan stating workdays and -weeks, off-days, deadlines, and meetings.

\section*{Abbreviations}
\begin{multicols}{2}
	\printacronyms[include-classes=abbrev, name=Abbreviations, heading=none]
\end{multicols}