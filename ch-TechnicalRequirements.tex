\chapter{Technical}

\section{Function requirements}
This section specifies the requirements set for the functions mentioned in figure \ref{fig:funcarch}.
\subsection{Generate flow}
In the project plan, a literature overview is given on perfusion phantoms, for a variety of organs, but also on physiological factors: perfusion rates, blood pressures, rates of stenosis et cetera. The TFR-GF requirements are based on the estimates by \cite{uren1994relation}, summarised in appendix \ref{app:physoverview}, \cite{chiribiri2013normal}, \cite{ho2014dynamic}, summarised in appendix \ref{app:physoverview_ho}, and \cite{slart2015pres}.

\begin{table} [H]
\caption{Function requirements for function: Generate flow}
\label{tab:funcreq_tec}
This table specifies the requirements for the generated flow and pressure.
\begin{tabular}{l|p{120mm}|}
	\makecell[l]{\textbf{Requirement} \\  \textbf{number}} & \multicolumn{1}{c}{\textbf{Description}}\\
	\hline
	TFR-GF0 & A constant flow is to be generated, i.e. non-pulsatile. \\
	TFR-GF0 & Flow generators need to be interchangeable. \\
	TFR-GF0 & Minimum achievable upper limit of myocardial perfusion is 300 mL/min/100g. \\
	TFR-GF0 & Minimum achievable lower limit of myocardial perfusion is 60 ml/min/100g. \\
	TFR-GF0 & Minimum achievable upper limit of cardiac output is 8 L/min.\\
	TFR-GF0* & Minimum arterial pressure is 56 mmHg. \\
	TFR-GF0* & Maximum arterial pressure is  155 mmHg. \\
	TFR-GF0 & Flow generators are controlled via a flow feedback system. \\
	\cline{2-2}
\end{tabular}
\raggedright
\textit{* based on the diastolic and systolic blood pressures.}
\end{table}

\subsection{Measuring flow and pressure}
\begin{table} [H]
\caption{Function requirements for function: Measure flow and pressure}
\label{tab:funcmeas}
This table specifies the requirements for the measuring of flow and pressure.
\begin{tabular}{l|p{120mm}|}
	\makecell[l]{\textbf{Requirement} \\  \textbf{number}} & \multicolumn{1}{c}{\textbf{Description}}\\
	\hline
	TFR-MFP0 & Flow measuring accuracy less than 5\%. \\
	TFR-MFP0 & Pressure measuring accuracy less than 5\%. \\
	TFR-MFP0 & Minimum absolute flow resolution of 1 mL/min. \\
	TFR-MFP0 & Minimum sampling rate of 100Hz. \\
	\cline{2-2}
\end{tabular}
\raggedright
\end{table}

\subsection{Simulate myocardial perfusion}
\begin{table} [H]
\caption{Function requirements for function: Simulate myocardial perfusion}
\label{tab:funcsim}
This table specifies the requirements specific for the phantom that simulates the myocardial perfusion.
\begin{tabular}{l|p{120mm}|}
	\makecell[l]{\textbf{Requirement} \\  \textbf{number}} & \multicolumn{1}{c}{\textbf{Description}}\\
	\hline
	TFR-SIM0 & An \ac{AIF} must be extractable from either the aorta or the left ventricle chamber.\\
	TFR-SIM0 & Stenotic arteries are mimicked in a physiological way by physically narrowing (or increasing flow resistance) of certain arteries. \\
	TFR-SIM0 & Different stenotic severity, should be possible by, for example, variable flow resistors or interchanging components. \\
	TFR-SIM0 & Multiple chambers, or areas, should be present, such that ischaemic and non-inschaemic tissue can be visualised simultaneously. Typical software divide the heart into 17 chambers.\\
	TFR-SIM0 & Phantom's compartment model should match the currently practised protocol. Does the tracer diffuse, is it trapped in tissue et cetera. \\
	\cline{2-2}
\end{tabular}
\raggedright
\end{table}

\subsection{Inject contrast}
The injection protocol is not part of the development of the phantom. However, there are certain requirements to be monitored:
\begin{table} [H]
\caption{Contrast requirements}
\label{tab:injcon}
This table summarises the requirements on the contrast and contrast injection protocol.
\begin{tabular}{l|p{120mm}|}
	\makecell[l]{\textbf{Requirement} \\ \textbf{number}} & \multicolumn{1}{c}{\textbf{Description}}\\
	\hline
	TR-IC0 & A variable amount of contrast can be injected. \\
	TR-IC0 & Contrast injection is reproducible. \\
	TR-IC0 & Contrast should match the currently practised protocol.  \\
	\cline{2-2}
\end{tabular}
\end{table}

\section{Physical requirements}
\rrot{Determine size of seating of D-SPECT}
\rrot{Determine weight limit of seating of D-SPECT}
Size, weight et cetera.
\begin{table} [H]
\caption{Physical requirements}
\label{tab:physrec}
This table summarises the physical requirements.
\begin{tabular}{l|p{120mm}|}
	\makecell[l]{\textbf{Requirement} \\ \textbf{number}} & \multicolumn{1}{c}{\textbf{Description}}\\
	\hline
	TR-PR0 & The phantom, and its set-up, must fit on the D-SPECT's chair. \\ 
	\cline{2-2}
\end{tabular}
\end{table}

\section{Environmental requirements}
\rrot{Determine how much noise output it may have.}
\rrot{Determine the height of the chair of the D-SPECT}
In what environment is the system operating.
\begin{table} [H]
\caption{Environmental requirements}
\label{tab:envirreq}
This table summarises the environmental requirements, i.e. the restrictions set by the environment to the phantom.
\begin{tabular}{l|p{120mm}|}
	\makecell[l]{\textbf{Requirement} \\ \textbf{number}} & \multicolumn{1}{c}{\textbf{Description}}\\
	\hline
	TR-ER0* &  No high-density or "High-Z" material is to be used.\\ 
	\cline{2-2}
\end{tabular}
\raggedright
\textit{* High-density and "High-Z" material, i.e. material with high atomic number, tend to block gamma radiation emitted by \ac{SPECT} tracers. Examples are Titanium (Ti), Chromium (Cr), Vanadium (V), Iron (Fe), or Lead (Pb); atom number \textgreater 22, Lead is 82.}
\end{table}

\section{External interfaces}
\begin{table} [H]
\caption{External interface requirements}
\label{tab:exint}
This table summarises the requirements for the external interface.
\begin{tabular}{l|p{120mm}|}
	\makecell[l]{\textbf{Requirement} \\ \textbf{number}} & \multicolumn{1}{c}{\textbf{Description}}\\
	\hline
	TR-EI0 &  Live plotting, at 10Hz, of system system flow and pressure.\\
	TR-EI0 & Ability to adjust the output of the flow generators. \\
	TR-EI0 & Serial communication between control/monitoring systems and external interface.\\
	\cline{2-2}
\end{tabular}
\end{table}

\section{System qualities}
\rrot{Specify pressure threshold.}
Define the quality of the system: such as reliability, availability, serviceability, security, scalability, maintainability.
\begin{table} [H]
\caption{System qualities}
\label{tab:sysqual}
This table summarises the system qualities.
\begin{tabular}{l|p{120mm}|}
	\makecell[l]{\textbf{Requirement} \\ \textbf{number}} & \multicolumn{1}{c}{\textbf{Description}}\\
	\hline
	TR-SQ0 & The flow set-up must perform an emergency shut down when the arterial pressure exceeds specified threshold. \\ 
	TR-SQ1 & The flow set-up must perform an emergency shut down when the flow cannot be controlled, i.e. erratic. \\
	\cline{2-2}
\end{tabular}
\end{table}

\section{Constraints and Assumptions}
Design constraints that have been imposed and assumptions that have been made by the requirements engineering team when gathering and analyzijng the requirements.

\begin{table}[H]
\caption{}
\label{tab:constassump}
This table summarises constraints placed on the design and assumptions made to yield the system requirements.
\begin{tabular}{l|p{120mm}|}
	\makecell[l]{\textbf{Requirement} \\ \textbf{number}} & \multicolumn{1}{c}{\textbf{Description}}\\
	\hline
	TR-CA0 &  Cardiac artefacts, beating of the heart, is initially too complex. The phantom will be static. \\
	TR-CA0 & Breating artifacts are not simulated in the phantom itself. A breathing thorax phantom can be used if available. \\
	TR-CA0 & Chest size, the amount of tissue between heart and scanner, is not simulated in the phantom itself. Thorax phantoms with modular rings are available to simulate tissue patients with varying BMIs. \\
	\cline{2-2}
\end{tabular}
\end{table}
