\chapter{Technical system overview}

\section{Function requirements}
This section specifies the requirements set for the functions mentioned in figure \ref{fig:funcarch}.
\subsection{Generate flow}
In the project plan, a literature overview is given on perfusion phantoms, for a variety of organs, but also on physiological factors: perfusion rates, blood pressures, rates of stenosis et cetera. The TFR-GF requirements are based on the estimates by \cite{uren1994relation}, summarised in appendix \ref{app:physoverview}, \cite{chiribiri2013normal}, \cite{ho2014dynamic}, summarised in appendix \ref{app:physoverview_ho}, and \cite{slart2015pres}.

Decisions and design choices are given in table \ref{tab:genflow_text}, quantitative requirements are given in table \ref{tab:genflow_quan}.

\begin{table}[H]
\caption{Textual requirements for function: Generate flow}
\label{tab:genflow_text}
\begin{tabular}{p{25mm}|p{115mm}|}
	\textbf{Requirement number} & \multicolumn{1}{c}{\textbf{Description}} \\
	\hline
	TFR-GFT01 & A variable, but constant, flow is to be generated, i.e. non-pulsatile. \\
	TFR-GFT02 & Flow generators need to be interchangeable. \\
	TFR-GFT03 & Flow feedback control for flow generators. \\
	\cline{2-2}
\end{tabular}
\end{table}

\textbf{TFR-GFT01} is based on reducing the complexity of the set-up. The ROI based AIF averages the intensity over time, which removes the pulsatile nature. Furthermore, the heart rate cannot be determine in the measurements results. Therefore, pulsatile flow is not a priority.
\textbf{TFR-GFT02} is based on maintaining flexibility such that the most optimal flow generator can be chosen based on the requirements for a specific experiment.

\textbf{TFR-GFT03} is based on ensuring reliability; no validation can be performed when the flow is not controlled.

\begin{table}[H]
\caption{Quantitative requirements for function: Generate flow}
\label{tab:genflow_quan}
\begin{tabular}{p{24mm}|p{65mm}ccp{21mm}|}
	\textbf{Requirement number} & \multicolumn{1}{c}{\textbf{Description}} & \multicolumn{1}{c}{ } & \multicolumn{1}{c}{\textbf{Value}} & \multicolumn{1}{c}{\textbf{Unit}} \\
	\hline
	TFR-GFQ01*	& Upper limit myocardial perfusion. 		 		& = 				& 300 				&  mL/min/100g \\
	TFR-GFQ02* 	& Lower limit myocardial perfusion. 				& = 				& 60 				& mL/min/100g \\
	TFR-GFQ03* 	& Typical perfusion rate during stress. 	 		& > \spacing < 		& 190 \spacing 300 	& mL/min/100g \\
	TFR-GFQ04*  	& Typical perfusion rate during rest. 			& > \spacing < 		& 60 \spacing 95 	& mL/min/100g \\
	TFR-GFQ05**	& Upper limit cardiac output.				 		& =					& 8 				& L/min \\
	TFR-GFQ06+		& Lower limit arterial pressure.				& =					& 56				& mmHg \\
	TFR-GFQ07+		& Upper limit arterial pressure.				& = 				& 155				& mmHg \\
	TFR-GFQ08		& Mean Arterial Pressure (MAP)\footnotemark. 	& = 				& 89				& mmHg \\
	TFR-GFQ09		& Typical MAP.								 	& > \spacing <		& \invchar 70 \spacing 110	& mmHg \\
	TFR-GFQ10 	& Feedback control accuracy 						& =					& 5					& \% \\
	\cline{2-5}
\end{tabular} \\
\raggedright
\textit{* combined flow to myocardium, indicated by blue arrows in figure \ref{fig:sim_heart}.} \\
\textit{** flow \textbf{not} entering the myocardium, indicated by red arrow in figure \ref{fig:sim_heart}.} \\
\textit{+ based on diastolic and systolic blood pressures, respectively. Measured at dashed line P in figure \ref{fig:sim_heart}.}
\end{table}

\footnotetext{Calculated as: $MAP \simeq DP + \sfrac{1}{3} (SP-DP)$}

\begin{figure}
\centering
\begin{minipage}{.5\textwidth}
  \centering
  \includegraphics[width=0.7\linewidth]{./images/simplified_heart.png}
  \captionof{figure}{Simplified, schematic overview of the heart.}
  \label{fig:sim_heart}
\end{minipage}%
\begin{minipage}{.5\textwidth}
  \centering
  \includegraphics[width=0.95\linewidth]{./images/17_segment.jpg}
  \captionof{figure}{17-segment heart model}
  \label{fig:segment_heart}
\end{minipage}
\end{figure}

\subsection{Measuring flow and pressure}
\begin{table}[H]
\caption{Quantitative requirements for function: Measure flow and pressure}
\label{tab:measflow_quan}
\begin{tabular}{p{25mm}|p{65mm}ccp{20mm}|}
	\textbf{Requirement number} & \multicolumn{1}{c}{\textbf{Description}} & \multicolumn{1}{c}{ } & \multicolumn{1}{c}{\textbf{Value}} & \multicolumn{1}{c}{\textbf{Unit}} \\
	\hline
	TFR-MFPQ01	& Flow measuring accuracy. 		 			 & <= 		& 5 		& \% \\
	TFR-MFPQ02 	& Pressure measuring accuracy.		 		 & <= 		& 5 		& \% \\
	TFR-MFPQ03 	& Absolute flow resolution.				 	 & >=	 	& 1 		& mL/min \\
	TFR-MFPQ04  & Sampling rate.					 		 & >= 		& 10 		& Hz \\
	\cline{2-5}
\end{tabular} \\
\raggedright
\end{table}

\begin{figure}[H]
	\includegraphics[width=0.5\linewidth]{./images/17_supply.png}
	\caption{Schematic representation of the supply to each segment (simplified).}
	\label{fig:segment_supply}
\end{figure}

\begin{figure}[H]
	\includegraphics[width=0.5\linewidth]{./images/17_segment_2.jpg}
	\caption{Schematic representation of the supply to each segment.}
	\label{fig:segment_supply2}
\end{figure}

\subsection{Simulate myocardial perfusion}
\begin{table} [H]
\caption{Function requirements for function: Simulate myocardial perfusion}
\label{tab:funcsim}
This table specifies the requirements specific for the phantom that simulates the myocardial perfusion.
\begin{tabular}{l|p{120mm}|}
	\makecell[l]{\textbf{Requirement} \\  \textbf{number}} & \multicolumn{1}{c}{\textbf{Description}}\\
	\hline
	\sout{TFR-SIMT01} & \sout{An \ac{AIF} must be extractable from the left ventricle, as per software requirement.}\\
	TFR-SIM02 & Stenotic arteries are mimicked in a physiological way by physically narrowing (or increasing flow resistance) of certain arteries. \\
	TFR-SIMT03 & Different stenotic severity, should be possible by, for example, variable flow resistors or interchanging components. \\
	TFR-SIMT04 & The phantom must be compatible with D-SPECT protocol. \\
	\hspace{1.5cm} A) & Flow to the myocardium is supplied by the RCA, LAD, and LCx. \\
	\hspace{1.5cm} B) & Flow for each segment is supplied individually by branches of the RCA, LAD, and LCx, see figure \ref{fig:segment_supply}. \\
	\hspace{1.5cm} C) & Flow from each segment is measured separately such that they can be compared to the 17-segment model. \\
	\hspace{1.5cm} D) & An ROI for the AIF can be taken in the left atrium. Alternatively, the ROI for the AIF can be taken in the left ventricle. \\
	\hspace{1.5cm} \sout{E)} & \sout{An AIF can be taken from the left atrium.} \\
	\hspace{1.5cm} F) & The left ventricle's myocardium has a Vertical and Horizontal Longitudinal Axial (VLA/HLA) cross-sectional shape of a horseshoe. \\
	\hspace{1.5cm} G) & The left ventricle's myocardium has a Short Axial (SA) cross-sectional shape of a circle. \\
	\hspace{1.5cm} H) & The phantom is oriented such that it mimics a patient in supine position. \\
	TFR-SIMT05* & Phantom's compartment model should match the currently practised protocol.\\
	\hspace{1.5cm} A) & The tracer specified in section \ref{sec:inj_tracer}.\\
	\hspace{1.5cm} B) & The contrast agent is absorbed by the myocardium to approximately 1.2\% of administered activity in 5 minutes. \\
	\hspace{1.5cm} C) & Contrast accumulates in skeletal muscles, spleen, liver, and kidneys (potential interference). \\
	\cline{2-2}
\end{tabular}
\raggedright
\textit{* \url{https://pubchem.ncbi.nlm.nih.gov/compound/131704316\#section=Absorption-Distribution-and-Excretion}}
\end{table}

\textbf{TFR-SIMT02} is based on the assumption that the relation between arteries, especially when some are narrowed, is too complex to be modelled independently. Simply reducing the overall flow in the myocardium will not capture that relation. Each segment of the left ventricle is supplied by a different branch of the three coronary arteries. One narrowed branch will have an impact on \textit{all} other branches, which leads to \textbf{TFR-SIMT03}. The severity of the stenosis will impact the other branches differently.

\textbf{TFR-SIMT03} is based on the goal of the project; to validate the D-SPECT. As mentioned in section \ref{sec:concept_oper}, the relatively less expensive, less invasive (patient friendliness and dose reduction), faster and more accurate system makes it suitable for myocardial perfusion imaging. However, the quantitative nature of the dynamic scanning protocol requires validation since it has not yet been done. Furthermore, the learning, educational, and training purposes of the phantom study is desired by researchers, manufacturers, and medical personnel. This is somewhat extended by \textbf{TFR-SIMT05}. Protocols already exist within clinics and is therefore the best starting point for research and phantom development.

\subsection{Inject tracer}
\label{sec:inj_tracer}
The injection protocol is not part of the development of the phantom. However, there are certain requirements to be monitored:

\begin{table}[H]
\caption{Textual requirements for function: Inject tracer}
\label{tab:injtrac_text}
\begin{tabular}{l|p{115mm}|}
	\makecell[l]{\textbf{Requirement} \\  \textbf{number}} & \multicolumn{1}{c}{\textbf{Description}} \\
	\hline
	TFR-ICT01 			& Tracer volume is variable. \\
	TFR-ICT02			& Tracer activity is variable, also see TFR-ICQ03. \\
	TFR-ICT03			& Tracer agent is variable. \\
	TFR-ICT04 			& Tracer injection is reproducible, also see TFR-ICT05.\\
	TFR-ICT05 			& Tracer protocol should match the currently practised protocol. \\
	\hspace{1.5cm} A) 	& See TFR-ICQ01. \\
	\hspace{1.5cm} B) 	& Tracer is injected, as bolus, via infusion pump. \\
	\hspace{1.5cm} C) 	& A pre-bolus is to precede the main bolus. \\
	\cline{2-2}
\end{tabular}
\end{table}

\textbf{TFR-ICT01} through \textbf{TFR-ICT03} are defined such that the tracer protocol can be optimised by performing experiments with different volumes, activity, or tracers. However, the first experiments will focus on the currently practised protocol, as is stated in \textbf{TFR-ICT05}. \textbf{TFR-ICT04} is based on the first experiments performed at the ZGT, Hengelo, where it is concluded that manual injection is not reproducible and results in unreliable results. These effect are directly visible in the dynamic scans. Therefore, an infusion pump is to be used.

\begin{table}[H]
\caption{Quantitative requirements for function: Inject tracer}
\label{tab:injtrac_quan}
\begin{tabular}{l|p{65mm}ccp{20mm}|}
	\makecell[l]{\textbf{Requirement} \\  \textbf{number}} & \multicolumn{1}{c}{\textbf{Description}} & \multicolumn{1}{c}{ } & \multicolumn{1}{c}{\textbf{Value}} & \multicolumn{1}{c}{\textbf{Unit}} \\
	\hline
	TFR-ICQ01 	& Tracer to be used. 			& = 			& \multicolumn{2}{p{35mm}|}{Technetium (\textsuperscript{99m}Tc) Tetrofosmin} \\
	TFR-ICQ02 	& Pre-bolus activity.			& = 			& 37				& Mega Becquerel \\
	TFR-ICQ03*	& Typical main bolus activity. 	& > \spacing < 	& 500 \spacing 700 	& Mega Becquerel \\
	\cline{2-5}
\end{tabular} \\
\raggedright
\textit{* hefty patient tend to get higher activity injected, i.e. 700 MBq.}
\end{table}

\section{Physical requirements}
\rrow{Determine size of seating of D-SPECT}
\rrod{Determine weight limit of seating of D-SPECT}
\rrot{Must it be completely anatomical?}
\rrot{Adjust requirements if the phantom does not have to be anatomical.}
The following requirements state the physical aspects of the phantom and of the .

\begin{table}[H]
\caption{Physical requirements (textual)}
\label{tab:physrec_text}
\begin{tabular}{l|p{115mm}|}
	\makecell[l]{\textbf{Requirement} \\  \textbf{number}} & \multicolumn{1}{c}{\textbf{Description}} \\
	\hline
	TR-PRT01 			& The phantom's left ventricle is to be placed inside the QRM TRX-116, see TR-PRQ01. \\
	TR-PRT02 			& The phantom's left ventricle must fit in the D-SPECT's imaging area. \\
	TR-PRT03 			& The phantom must be anatomically shaped. \\
	\hspace{1.5cm} \sout{A)} 	& \sout{In correspondence with requirements TFR-SIMT04.} \\
	\hspace{1.5cm} B) 	& Four chambered phantom that correspond to left/right ventricle and left/right atrium. \\
	\hspace{1.5cm} C) 	& Segmented myocardium surrounds heart chambers. \\
	\hspace{1.5cm}\sout{D)}	& \sout{Three coronary arteries, RCA, LAD and LCx, supply the myocardium.} \\
	\hspace{1.5cm} E) 	& The coronary arteries run outside of the myocardium. \\
	\hspace{1.5cm} F) 	& The coronary veins run outside of the myocardium. \\
	TR-PRT04 			& The flow set-up is to remain horizontal (preventing additional flow resistance). \\
	TR-PRT05 			& The phantom cannot contain air bubbles. \\
	\cline{2-2}
\end{tabular}
\end{table}

\rrot{Why only left ventricle?}
\textbf{TR-PRT01} is based on creating realistic simulation of myocardial perfusion, thereby requiring a thorax phantom (with possible extension rings to simulate more hefty patients). The QRM TRX-116 has been successfully used for CT experiments. The 4DM software looks at the left ventricle thereby requiring the left ventricle to be in the phantom and in the imaging area, as stated in \textbf{TR-PRT02}.

\rrot{Must it be anatomically shaped?}

\textbf{TR-PRT04} is based on the choice to prevent unnecessary complexity. Remaining horizontal will negate gravity.

\textbf{TR-PRT05} is based on the attenuation of air, which compromises the TAC determination.

\begin{table}[H]
\caption{Physical requirements (Quantitative)}
\label{tab:physrec_quan}
\begin{tabular}{l|p{65mm}ccp{20mm}|}
	\makecell[l]{\textbf{Requirement} \\  \textbf{number}} & \multicolumn{1}{c}{\textbf{Description}} & \multicolumn{1}{c}{ } & \multicolumn{1}{c}{\textbf{Value}} & \multicolumn{1}{c}{\textbf{Unit}} \\
	\hline	
	TR-PRQ01 & Short Axial diameter.		 						& < 			& 100 							& Millimetre \\
	TR-PRQ02 & Weight on patient chair. 							& < 			& 171 							& Kilogram   \\
	TR-PRQ03+ & Phantom's outer dimensions. 							& 				& 								& 			 \\
	\hspace{1.5cm} A) & Basal-Apical distance. 						& $\approx$ 	& 120 							& Millimetre \\
	\hspace{1.5cm} B) & Left-Right Lateral distance.				& $\approx$ 	& 80							& Millimetre \\
	\hspace{1.5cm} C) & Anterior-Posterior distance. 				& $\approx$ 	& 60							& Millimetre \\
	TR-PRQ04++ & Left ventricle dimensions.							& 				& 								& 			 \\
	\hspace{1.5cm} A)* & Internal Apical-Annular distance.			& > \spacing < 	& \invchar 69.4 \spacing 105.8	& Millimetre \\
	\hspace{1.5cm} B) & Internal Septal-Lateral distance. 			& > \spacing <	& 38.2 \spacing 55.6			& Millimetre \\
	\hspace{1.5cm} C) & Internal Anterior-Inferior.					& > \spacing < 	& 46.9 \spacing 68.5 			& Millimetre \\
	\hspace{1.5cm} D) & Myocardial wall thickness.					& > \spacing < 	& 4.8 \spacing 9.8				& Millimetre \\
	\hspace{1.5cm} E)= & Internal volume.							& > \spacing < 	& \invchar 47 \spacing 156 	& Millilitre \\
	TR-PRQ05++ & Right ventricle dimensions.							& 				&								&			 \\
	\hspace{1.5cm} A) & Internal Apical-Annular distance.			& > \spacing <	& 44.8 \spacing 79.2 			& Millimetre \\
	\hspace{1.5cm} B) & Internal Septal-Medial	distance.			& > \spacing < 	& 19.2 \spacing 40.0 			& Millimetre \\
	\hspace{1.5cm} C) & Internal Anterior-Inferior distance.		& > \spacing < 	& 42.2 \spacing 73.6 			& Millimetre \\
	\hspace{1.5cm} D) & Myocardial wall thickness.					& > \spacing <	& 1.0 \spacing 3.8				& Millimetre \\
	\hspace{1.5cm} E)= & Internal volume. 							& > \spacing <	&  \invchar 24.9 \spacing 163.0 & Millilitre \\
	TR-PRQ06+ 	& Phantom resembles weight of average human heart. & > \spacing < & 250 \spacing 350 & Gram \\
	\cline{2-5}
\end{tabular} \\
\raggedright
\textit{* Annular $\rightarrow$ Annulus $\rightarrow$ assuming mitral valve level.} \\
\textit{+ based on \cite{openstax2013anatomy}.} \\
\textit{++ based on \cite{lin2008cardiac}.} \\
\textit{= based on \cite{maceira2006normalizedleft} and \cite{maceira2006normalizedright}}
\end{table}

\section{Environmental requirements}
\rrot{Determine how much noise output it may have.}
\rrod{Determine the height of the chair of the D-SPECT}
In what environment is the system operating.
\begin{table} [H]
\caption{Environmental requirements (Textual)}
\label{tab:envirreq_text}
\begin{tabular}{l|p{120mm}|}
	\makecell[l]{\textbf{Requirement} \\ \textbf{number}} & \multicolumn{1}{c}{\textbf{Description}}\\
	\hline
	TR-ERT01 & No high-density or "High-Z" material is to be used.\\ 
	TR-ERT02 & The phantom's left and front side must remain free, see figure \ref{fig:spect_surround}. \\ 
	\sout{TR-ERT03**} & \sout{Any part of the flow set-up and/or phantom, that does not fit directly on the patient chair, must remain horizontal with the remaining parts between 63 and 93cm.} \\
	\cline{2-2}
\end{tabular}
\raggedright

\end{table}

\begin{figure}
  \includegraphics[width=0.5\linewidth]{./images/surrounding_spect.jpg}
  \caption{figure}{Schematic drawing of D-SPECT head\citep{erlandsson2009performance}.}
  \label{fig:spect_surround}
\end{figure}

\textbf{TR-ERT01} is based on material properties; "High-Z", or High-Density, material tend to block gamma radiation emitted by \ac{SPECT} tracers. Some examples of High-Z materials are Titanium (Ti), Chromium (Cr), Vanadium (V), Iron (Fe), or Lead (Pb).

\textbf{TR-ERT02} is based on the D-SPECTS design. The curved design allows for better patient comfort and proper imaging, but will require the phantom for being accessible, i.e. not blocked by High-Z materials, from the patient's left and front side.

\begin{table}[H]
\caption{Environmental requirements (Quantitative)}
\label{tab:envirreq_quan}
\begin{tabular}{l|p{65mm}ccp{20mm}|}
	\makecell[l]{\textbf{Requirement} \\  \textbf{number}} & \multicolumn{1}{c}{\textbf{Description}} & \multicolumn{1}{c}{ } & \multicolumn{1}{c}{\textbf{Value}} & \multicolumn{1}{c}{\textbf{Unit}} \\
	\hline	
	TR-ERQ01* &  Electric power. &  &  &  \\
	\hspace{1.5cm} A) & Supply voltage. 	& = 	& 230 	& Volt \\
	\hspace{1.5cm} B) & Supply current at TR-ERQ01 A). 	& < < 	& 6		& Ampere \\
	\hspace{1.5cm} C) & Supply type. 		& = 	& AC 	& - \\
	\hspace{1.5cm} C2) & Supply frequency	& =		& 50	& Hertz \\
	\cline{2-5}
\end{tabular} \\
\raggedright
\textit{* electric power connection (wall socket) for all systems, standard Dutch power mains. \textbf{No more than TR-ERQ01 B) can be drawn due to hospital safety measures.}}
\end{table}

\section{External interfaces}
\begin{table} [H]
\caption{External interface requirements (textual)}
\label{tab:extint_text}
\begin{tabular}{l|p{120mm}|}
	\makecell[l]{\textbf{Requirement} \\ \textbf{number}} & \multicolumn{1}{c}{\textbf{Description}}\\
	\hline
	TR-EIT01 & Adjust output of flow generators. \\
	TR-EIT02 & Serial communication between control/monitoring systems and external interface.\\
	\cline{2-2}
\end{tabular}
\end{table}

\textbf{TR-EIT01} is based on the different experiments that need to be performed at different flow rates to determine the effect on the outcome.

\textbf{TR-EIT02} is based on the current control and monitoring system, which is connected via USB to the external interface running on in MATLAB on a laptop.

\begin{table}[H]
\caption{External interface requirements (Quantitative)}
\label{tab:extint_quan}
\begin{tabular}{l|p{65mm}ccp{20mm}|}
	\makecell[l]{\textbf{Requirement} \\  \textbf{number}} & \multicolumn{1}{c}{\textbf{Description}} & \multicolumn{1}{c}{ } & \multicolumn{1}{c}{\textbf{Value}} & \multicolumn{1}{c}{\textbf{Unit}} \\
	\hline	
	TR-EIQ01 &  Live plotting frequency of system's flow and pressure. & = &  10 & Hertz \\
	\cline{2-5}
\end{tabular} \\
\end{table}

\section{System qualities}
\rrot{Specify pressure threshold.}
Define the quality of the system: such as reliability, availability, serviceability, security, scalability, maintainability.
\begin{table} [H]
\caption{System qualities}
\label{tab:sysqual}
\begin{tabular}{l|p{120mm}|}
	\makecell[l]{\textbf{Requirement} \\ \textbf{number}} & \multicolumn{1}{c}{\textbf{Description}}\\
	\hline
	TR-SQT01 & Emergency shut down of flow set-up when arterial pressure exceeds TFR-GFQ07. \\ 
	TR-SQT02 & Emergency shut down of flow set-up when flow cannot be controlled, i.e. erratic or absent. \\
	TR-SQT03 & No reversed flow out of the phantom is allowed. \\
	\cline{2-2}
\end{tabular}
\end{table}

\textbf{TR-SQT01} and \textbf{TR-SQT02} are based on safety and prevention of leakage. Excessive pressure indicates faulty situation which must be resolved before components fail. Erratic, and especially the absence of proper flow, indicates a leakage and must be resolved. Leakage after injecting the tracer must be prevented at all costs.

\textbf{TR-QRT03} is based on optimisation of the experiments. Once the phantom is filled, it must remain filled such that experiments can be performed quickly in succession. 

\section{Constraints and Assumptions}
Design constraints that have been imposed and assumptions that have been made by the requirements engineering team when gathering and analyzijng the requirements.

\begin{table}[H]
\caption{}
\label{tab:constassump}
This table summarises constraints placed on the design and assumptions made to yield the system requirements.
\begin{tabular}{l|p{120mm}|}
	\makecell[l]{\textbf{Reference} \\ \textbf{number}} & \multicolumn{1}{c}{\textbf{Description}}\\
	\hline
	TR-CAT01 & Beating artefacts will not be generated. \\
	TR-CAT02 & Breathing artefacts will not be generated. \\
	TR-CAT03 & Hefty patients are simulated using extension rings on the thorax phantom.  \\
	\cline{2-2}
\end{tabular}
\end{table}

\textbf{TR-CAT01} and \textbf{TR-CAT02} are set to prevent over-complicating the first myocardial perfusion phantom. Breathing artefacts may be generated by means of a breathing thorax phantom, which is being developed in Munster, Germany. However, it will make the first phantom too complex but can potentially be used for the second iteration.

Extension rings can be used for the static thorax phantom, see TR-PRT01. These extension rings can increase the amount of "tissue" between the heart phantom (placed in the center) and the scanner. This will simulate more hefty patients, as stated by \textbf{TR-CAT03}.