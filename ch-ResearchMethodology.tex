\chapter{Research methodology}
This chapter serves as a summary of the previously answered research questions and gives an overview of the research questions to come.
\section{Main research question}
\textit{Can patient treatment reliably depend on the D-SPECT, using dynamic scanning, in myocardial perfusion imaging?}
\subsection*{Answer}
As of \today, the main research question has not been answered.
\subsection*{Answered in}
It will be answered in the final report of the master's thesis.
\subsection*{Based on}
The answer will be based on the developed myocardial perfusion phantom and the experiments performed with it at the ZGT Hengelo.
\section{Concept of Operations}
\label{sec:concept_oper}
\textit{Is the D-SPECT's dynamic scanning, in comparison with other modalities(CT, MRI, PET, or SPECT), suitable for quantitative perfusion imaging?}

\textit{What must the myocardial perfusion phantom be able to simulate?}
\subsection*{Answer}
The D-SPECT is relatively new in the Netherlands, but it is more widely employed in Japan, Canada, France, and Great-Britain. The highly specialised nature (for cardiac purposes), the patient friendly design, the ability to scan faster and more accurate at significant dose reductions, make the D-SPECT suitable for quantitative myocardial perfusion.

The myocardial perfusion phantom will have to simulate a patient, with stenotic artery (or arteries), in a physiological way, which is compatible with clinical protocol and software.
\subsection*{Answered in}
The answer to this research question can be found in the system requirements document section 2.2.1 and 2.2.2, respectively.
\subsection*{Based on}
The answer to this research question is based on the literature review and background investigation performed in the project plan, chapter 2.
\section{Requirements and Architecture}
What are the requirements for a myocardial perfusion phantom that can be used in combination with commonly used clinical software?
\subsection*{Answer} 
The requirements are specified in tables corresponding in the system requirements document.
\subsection*{Answered in}
The answer to this research question can be found int he system requirements document, chapters 2  and 3.
\subsection*{Based on}
The answer to this research question is based on interviews with a part of the direct stakeholders, as specified in the project plan section 3.2.
\section{Detailed Design}
\textit{How can the myocardial perfusion phantom meet the clinical requirements and mimic the perfusion of a human heart?}
\subsection*{Answer}
It will be answered in the this detailed design document.
\subsection*{Answered in}
The answer can be found in chapter \ref{ch:detailed_design}.
\subsection*{Based on}
The answer to this research question is based on a mind map which results in different concept. The most promising concept, based on the requirements, is developed further into a detailed design. 